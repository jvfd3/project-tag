\chapterimage{Pictures/Ausberto/10_Conclusao.png}

\chapter{Considerações Finais}

Os problemas enfrentados no desenvolvimento do projeto Tag (Figura \ref{Símbolo}) foram principalmente dois: em primeiro lugar a gestão de tempo associada a falta de priorização da conclusão das tarefas propostas e em segundo lugar a tentativa de se aprender a linguagem de programação TypeScript em conjunto com a biblioteca  sem começar o aprendizado pela base. O primeiro problema causou um desgaste muito maior do que o necessário, visto que o tempo poderia ter sido muito melhor utilizado para entregar um trabalho muito melhor sem preocupação com o prazo. O segundo, tendo relação com o primeiro, caso o desenvolvimento houvesse começado com mais antecedência, haveria tempo hábil o bastante para que fosse inicialmente estudada a linguagem e só em seguida o código fosse desenvolvido.

O trabalho foi desenvolvido utilizando a programação orientada a objetos, metodologia que por fim, mais pareceu atrapalhar do que colaborar com o desenvolvimento. Parte porque o uso de componentes de classe em React estarem entrando em desuso e dando espaço para componentes funcionais; parte porque, não foi aprendido previamente quanto a como desenvolver apropriadamente um sistema orientado a objetos.

Analisando em retrospecto, vê-se que um maior preparo e maior antecedência seriam as melhores abordagens. Quanto ao preparo, seria adequado um maior estudo sobre o uso da orientação a objetos na prática utilizando a linguagem escolhida. E de um modo geral, um aprendizado mais profundo e prévio em relação ao funcionamento da linguagem escolhida. Outra vertente também seria focar inicialmente na parte visual do código antes de desenvolver suas funcionalidades.

% Funcionalidades a serem desenvolvidas no futuro...

\begin{figure}[H]
    \begin{center}
        \includegraphics[width=12cm]{Pictures/JV/10_TagLogo.png}
        \caption{Símbolo do site} \label{Símbolo}
    \end{center}
\end{figure} 

\begin{comment}
    Prof. Dr. Ausberto S. Castro Vera
    UENF - CCT - LCMAT - Curso de Ciência da Computação
    Campos, RJ, 2022 
    Disciplina: Paradigma de Desenvolvimento Orientado a Objetos
    Aluno: João Vítor Fernandes Dias
\end{comment}