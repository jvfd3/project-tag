\chapterimage{sistemas.png} % Table of contents heading image
\chapter{ Introdução}

    % Prof. Dr. Ausberto S. Castro Vera
    % UENF - CCT - LCMAT - Curso de Ciência da Computação
    % Campos, RJ,  2022
    % Disciplina: Paradigma de Desenvolvimento Orientado a Objetos
    % Aluno: João Vítor Fernandes Dias

    \textit{Paradigma de Desenvolvimento de Sistemas Orientado a Objetos} é uma disciplina orientada a desenvolver um sistema utilizando  a metodologia orientada a Objetos em todas as etapas do Ciclo de Vida de Desenvolvimento de um Sistema (CVDS).  As referências bibliográficas básicas a serem consultadas são: \cite{Dennis2014}, \cite{Engholm2013}, \cite{Guedes2011},  \cite{Sommerville2018} e \cite{Wazlawick2011}. Como bibliografia complementar serão considerados: \cite{Satzinger2012}, \cite{Shelly2012} e  \cite{Furgeri2013}.
    
    Neste documento serão apresentadas as principais atividades realizadas para o desenvolvimento COMPLETO de uma aplicação OO.
    
    O sistema a ser desenvolvido é um website que poderá ser utilizado por coordenadores de cursos de ensino onde poderão cadastrar a disponibilidade dos professores, salas, alunos e matérias. O website retornará a ele uma organização inicial válida de organização do quadro de horários.

   \section{Contextualização}

        Considerando a atual situação de instituições de ensino superior que não contam com suporte técnico especializado suficiente para poder realizar suas tarefas de forma eficiente, nota-se uma interessante área de defasagem que pode ser abordada através de métodos heurísticos para a resolução de problemas. A área em questão é a seção otimização da organização do quadro de horários, organização esta proposta pelo presente projeto.

        \subsection{Aprofundamento do problema}

            Semestre após semestre, ano após ano, os coordenadores precisam demandar grande tempo e esforço para conseguir encaixar bem as matérias, professores, alunos e salas simultaneamente. Ou em uma situação ainda pior, não se esforçam e apenas deixam a grade ficar de qualquer jeito, sem forma alguma de otimização.

        \subsection{Soluções}

            Para que essa questão deixe de ser um problema, existem algoritmos heurísticos que auxiliam na busca pela resposta ótima, mesmo que seja computacionalmente custosa. Este sistema busca ser uma interface entre o coordenador e todo o mecanismo que de fato calculará o custo.

   \section{Objetivo do Sistema} % Um objetivo Geral e 3 objetivos específicos
   
        Os objetivos do sistema serão enumerados abaixo:
        
        \begin{enumerate}
            \item O sistema busca permitir o acesso a um banco de dados para o qual poderão ser enviados os extratos dos alunos;
            \item O sistema visa auxiliar os alunos a conseguirem uma grade de horários mais organizada, enxuta e que atenda às suas necessidades de matérias;
            \item Outra meta do sistema é poder gerar um relatório detalhado dos benefícios da solução encontrada por ele;
            \item Mas como principal objetivo do sistema está a redução do gasto de tempo dispendido em tarefas mecânicas e manuais necessárias por todos os gestores de profissionais do ensino.
        \end{enumerate}

   \section{Justificativa} % Justificar por que foi escolhido ou deve ser desenvolvido este sistema

        Visto a ausência de alternativas que solucionem o problema de gestão de demanda de matérias, professores e alunos, levando em conta também o custo de tempo necessário para tentar alcançar alternativas que nem sempre são as melhores, veem-se claros benefícios para o uso do sistema em questão.