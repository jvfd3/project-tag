\chapterimage{Pictures/Ausberto/09_Implementacao.jpg}

\chapter{Implementação do Sistema OO} % Fazer listagens

Neste capítulo serão apresentados fragmentos do código utilizado na implementação. Também serão mostradas as classes utilizadas e o processo necessário para a instalação do sistema.

\section{Programação}
    Listar:
    \begin{itemize}
        \item Todas as classes implementadas na linguagem OO (sem os métodos)
                
            \begin{enumerate}
                \item TagWholePage
                \item HeaderComponentClass
                \item PageTitle
                \item HandmadeRouter
                \item MainPage
                \item AddTagPage
                \item TagCreationHolder
                \item SelfPage
                \item SelfImage
                \item SelfUserContent
                \item AboutPage
                \item SignInPage
                \item UserSignForm
                \item SearchTagPage
                \item TagTable
                \item FooterComponentClass
            \end{enumerate}

        \item Todos os módulos ou subsistemas (diagramas ou arquitetura implementada)

            Cada classe foi colocada em um módulo diferente que atua semi-independentemente dos outros módulos. Assim, poderia-se dizer que os módulos seriam aqueles já mostrados na enumeração \ref{Classes}.
    
            
        \item Partes importantes do código fonte implementado
        
            \lstinputlisting[language=JavaScript, caption=Cabeçalho, label={code0}]{Codes/Components/header.tsx}

                No código \ref{code0} é criado o cabeçalho e seus botões.
            
            \lstinputlisting[language=JavaScript, label={code1}, caption=Sobre]{Codes/Pages/about.tsx}

                No código \ref{code1} é criada a página sobre, dispondo do símbolo do curso de computação e de informações pertinentes.
            
            \lstinputlisting[language=JavaScript, label={code2}, caption=Adicionar Tags]{Codes/Pages/add.tsx}

                No código \ref{code2} é desenvolvida a página referente à adição de tags.
            
            \lstinputlisting[language=JavaScript, label={code3}, caption=Pesquisa]{Codes/Pages/search.tsx}

                No código \ref{code3} é criada a página que exemplifica a pesquisa por tags já criadas.
            
            \lstinputlisting[language=JavaScript, label={code4}, caption=Informações do usuário]{Codes/Pages/self.tsx}

                No código \ref{code4} é criada a página que exemplifica a apresentação dos dados do usuário.
            
            \lstinputlisting[language=JavaScript, label={code5}, caption=Formulário de login]{Codes/Pages/signIn.tsx}

                No código \ref{code5} é criada a página de login e cadastro.
            
            \lstinputlisting[language=JavaScript, label={code6}, caption=Sub formulário de login]{Codes/Components/form.tsx}

                No código \ref{code6} é criado o formulário necessário para login e cadastro.
            
        \item As bases de dados implementadas
            \lstinputlisting[language=JavaScript, label={code7}, caption=Banco de dados de tags]{Codes/Databases/db_tags.json}

                O código \ref{code7} é referente ao banco de dados base utilizado referente às tags.
            
            \lstinputlisting[language=JavaScript, label={code8}, caption=Banco de dados de usuários]{Codes/Databases/db_username_password.json}

                O código \ref{code8} é referente ao banco de dados base utilizado para armazenar as contas de usuários.
            
        
    \end{itemize}

\section{Documentação do Software}

    Indicar e explicar:
    \begin{itemize}
        \item \textbf{Manual de Instalação}: O que é necessário para instalar o sistema desenvolvido (banco de dados, versão, bibliotecas, etc.)
        
            Para fazer a instalação, primeiro é necessário fazer a instalação do NodeJS pelo \href{https://nodejs.org/en/}{\textbf{site oficial}} ou diretamente através \href{https://nodejs.org/dist/v18.12.1/node-v18.12.1-x64.msi}{\textbf{deste link}}.
            
            \begin{enumerate}
                \item execute o arquivo.
                \item Clique em Next
                \item Leia os termos
                \item Clique na checkbox caso concorde com os termos
                \item Clique em Next
                \item Clique em Next
                \item Clique em Next
                \item Clique em Next
                \item Clique em Install
                \item Caso apareça um aviso do Windows, Clique em Sim
                \item Clique em Finish
            \end{enumerate}
    
            Agora, quanto ao código:
            \begin{enumerate}
                \item Faça o download dos arquivos contidos \href{https://github.com/jvfd3/project-tag/archive/refs/heads/main.zip}{neste link}
                \item extrai os arquivos para uma pasta de preferência
                \item Navegue pelas pastas: project-tag
                \item Ao entrar na pasta tag, pressione shift e clique com botão direito na pasta 
                \item Selecione a opção "abrir botão do power shell"
                \item ao abrir a tela, execute o comando "npm install"
                \item Depois execute o comando "npm start"
            \end{enumerate}

            Se tudo der certo, dentro de alguns instantes uma página do navegador aparecerá para que você possa mexer no sistema.
        \begin{comment}
            
        \item \textbf{Manual do usuário}: Os passos básicos para utilizar o sistema (inicializar, salvar, imprimir, etc.)
        \item \textbf{Outros}....
        \end{comment}
    \end{itemize}

\begin{comment}
    Prof. Dr. Ausberto S. Castro Vera
    UENF - CCT - LCMAT - Curso de Ci\^{e}ncia da Computação
    Campos, RJ, 2022 
    Disciplina: Paradigma de Desenvolvimento Orientado a Objetos
    Aluno: João Vítor Fernandes Dias
\end{comment}