\chapterimage{Pictures/Ausberto/06_Requisitos.jpg}

\chapter{Requisitos do Sistema OO}

Neste capítulo é apresentado listas, definições e especificações de Requisitos do sistema ser desenvolvido. Os requisitos são declarações abstratas de alto nível sobre os \textit{serviços} que o sistema deve prestar à organização, e as \textit{restrições} sobre as quais deve operar. Os requisitos sempre refletem as necessidades dos clientes do sistema.

Sobre os requisitos, Raul S. Wazlawick afirma:

\begin{citadireta}
    A \textit{etapa de levantamento de requisitos} corresponde a buscar todas as informações possíveis sobre as funções que o sistema deve executar e as restrições sobre as quais o sistema deve operar. O produto dessa etapa será o documento de requisitos, principal componente do anteprojeto de software.
    
    A \textit{etapa de análise de requisitos} serve para estruturar e detalhar os requisitos de forma que eles possam ser abordados na fase de elaboração para o desenvolvimento de outros elementos como casos de uso, classes e interfaces.
    
    O levantamento de requisitos é o processo de descobrir quais são as \textit{funções} que o sistema deve realizar e quais são as \textit{restrições} que existem sobre estas funções \cite{Wazlawick2011}.
\end{citadireta}

\section{Requisitos Funcionais}

    Os requisitos funcionais representam quais são as características e ações disponíveis pelo sistema para que ele opere como esperado.

    \subsection{Subsistema de Cadastro}
        Aqui estão listadas todas as atribuições que o usuário possui referente ao cadastro. 
        % Ator: usuário.
        % Ponto de partida: Tela de cadastro.
        \begin{itemize}
            \item Acessar o site
            \item Selecionar a caixa de texto referente ao nome de usuário
            \item Digitar nome de usuário
            \item Selecionar a caixa de texto referente à senha
            \item Digitar a senha
            \item Clicar em Salvar
            \item Cadastrar conta
        \end{itemize}
    \subsection{Subsistema de Login}
        Aqui estão listadas todas as atribuições que o usuário possui referente ao login.
        % Ator: usuário.
        % Ponto de partida: Tela de cadastro.
        \begin{itemize}
            \item Acessar o site
            \item Selecionar a caixa de texto referente ao nome de usuário
            \item Digitar nome de usuário
            \item Selecionar a caixa de texto referente à senha
            \item Digitar a senha
            \item Logar na conta
        \end{itemize}
    \subsection{Subsistema de Criação de Tags}
        Aqui estão listadas todas as atribuições que o usuário possui referente à criação de tags.
        % Ator: usuário.
        % Ponto de partida: Tela principal.
        \begin{itemize}
            \item Selecionar a caixa de texto referente a chave da tag
            \item Digitar a chave da tag
            \item Selecionar a caixa de texto referente o valor da tag
            \item Digitar o valor da tag
            \item Clicar em confirmar criação
        \end{itemize}
    \subsection{Subsistema de Busca de Tags}
        Aqui estão listadas todas as atribuições que o usuário possui referente à busca de tags.
        % Ator: usuário.
        % Ponto de partida: Tela principal.
        \begin{itemize}
            \item Clicar na barra de busca de tags
            \item Digitar chave da tag a ser buscada
            \item Clicar no botão para buscar a tag digitada
        \end{itemize}

\section{Requisitos Não-Funcionais} % Revisar referência % Adicionar textinho para cada subseção

    Os requisitos funcionais representam as características mais abstratas que atuam não diretamente em relação às funções do sistema.
    
    \subsection{Requisitos de Usabilidade}
        \begin{itemize}
            \item O sistema deve apresentar menos de 10 botões por página
            \item O sistema deve contar com uma paleta de cores tendendo ao preto
            \item Todos os botões devem ser facilmente identificáveis
        \end{itemize}
    \subsection{Requisitos de Confiabilidade}
        \begin{itemize}
            % \item Caso haja falha de conexão, o usuário precisa ser informado na mesma hora
            % \item Tudo que o usuário fizer no modo offline deverá ser guardado para ser executado quando retornar à conexão
            \item O usuário deve ser informado caso alguma de suas ações não tiverem tido efeito
        \end{itemize}
    \subsection{Requisitos de Disponibilidade}
        \begin{itemize}
            \item Durante a execução do website, ele não deve apresentar falhas catastróficas
            % \item O sistema deve estar online o tempo todo
            % \item Rotinas de checagem de falhas e erros devem rodar continuamente em busca de problemas
            % \item Em caso de falha catastrófica, os usuários devem ser notificados e o sistema deve voltar ao ar em menos de uma hora
        \end{itemize}
    \subsection{Requisitos de Privacidade}
        \begin{itemize}
            \item Todos os dados inseridos pelo usuário na plataforma poderão ser utilizados pelo sistema como bem entender, dados os limites impostos pelas leis
            \item O Sistema não pode ultrapassar os direitos do uso dos dados dos países cujas leis de dados o afetem
            % \item O usuário poderá ter a opção de não ceder dado algum
        \end{itemize}
    \subsection{Requisitos de Acesso}
        \begin{itemize}
            \item Mesmo usuários não cadastrados poderão interagir com o sistema de tags
            \item Gestores do sistema poderão manualmente apagar tags que foram criadas inadequadamente
            \item Gestores do sistema poderão manualmente criar novas tags
        \end{itemize}

 \section{Requisitos de Negócios}
     Requisitos do negócio são requisitos de alto nível que explicam e justificam qualquer projeto. Os requisitos de negócios são as atividades críticas de uma empresa que devem ser executadas para atender ao(s) objetivo(s) organizacional(is) enquanto permanecem independentes do sistema solução.

    \begin{itemize}
        \item Proporcionar aos usuários uma forma estruturada de acessar informações e filtrar diversos objetos que sejam de seu interesse
        % \item Tornar o conceito de tags natural e intuitivo para os usuários do sistema
        \item Ser capaz de comportar novas tags únicas, dados os limites de armazenamento de dados
    % e novas associações de dados 
        % \item Mesmo durante manutenções o sistema deve manter ativo
    \end{itemize}

\begin{comment}
    Prof. Dr. Ausberto S. Castro Vera
    UENF - CCT - LCMAT - Curso de Ciência da Computação
    Campos, RJ, 2022 
    Disciplina: Paradigma de Desenvolvimento Orientado a Objetos
    Aluno: João Vítor Fernandes Dias
\end{comment}