\chapterimage{Pictures/Ausberto/05_Sistemas.png}
% Pesquisar referências sobre JSON e Big Data e dados estruturados
\chapter{Introdução}

\textit{Paradigma de Desenvolvimento de Sistemas Orientado a Objetos} é uma disciplina orientada a desenvolver um sistema utilizando a metodologia orientada a Objetos em todas as etapas do Ciclo de Vida de Desenvolvimento de um Sistema (CVDS). As referências bibliográficas básicas a serem consultadas são: \cite{Dennis2014}, \cite{Engholm2013}, \cite{Guedes2011}, \cite{Sommerville2018} e \cite{Wazlawick2011}. Como bibliografia complementar serão considerados: \cite{Satzinger2012}, \cite{Shelly2012} e \cite{Furgeri2013}.

Neste documento serão apresentadas as principais atividades realizadas para o desenvolvimento COMPLETO de uma aplicação OO.

O sistema a ser desenvolvido é um website para adição e pesquisa de tags. Entende-se por "tag" (etiqueta em inglês), características e propriedades dadas a um determinado objeto.

\section{Contextualização}

    Já é um fato que uma enorme quantidade de dados trafega pela internet diariamente. Outra enorme quantidade de dados se encontra armazenada em diversos tipos de bancos de dados. Para acessá-los pela internet, necessitamos de indexadores e motores de busca que vasculham esses dados armazenados e tentam nos retornar o que queremos. Entretanto, nem sempre conseguimos encontrar facilmente o que buscamos. Não seria melhor se conseguíssemos pesquisar diretamente o que desejamos sem que nossa busca fosse automaticamente desviada?

    \subsection{Aprofundamento do problema}

        Ao tentar buscar por algum produto para comprar na internet, talvez você já tenha se perguntado algo como "Será que esse é o melhor produto?", "Será que há um outro produto desse que tenha a cor cinza?", "Quais outros produtos similares a esse existem?". Acaba sendo necessária uma busca extenuante para tentar coletar manualmente todas essas informações.

    \subsection{Soluções}

        Para que essa questão deixe de ser um problema, existem algoritmos de rastreamento de dados que podem vasculhar a internet em busca destes dados, estruturá-los, e permitir que o usuário busque de forma mais simples e direta por estas informações. Outro meio é através da colaboração dos próprios usuários que poderão manualmente adicionar quaisquer objetos com suas respectivas características. Este sistema busca ser uma interface em que os usuários poderão adicionar manualmente nomes de atributos e seus respectivos valores.

\section{Objetivo do Sistema} % Um objetivo Geral e 3 objetivos específicos

    Os objetivos do sistema serão enumerados abaixo:
    
    \begin{enumerate}
        \item O sistema busca permitir a criação de usuários;
        \item O sistema visa facilitar a organização de dados de forma coesa;
        \item Mas como principal objetivo do sistema está a redução da complexidade para que usuários comuns encontrem o que procuram.
    \end{enumerate}

\section{Justificativa} % Justificar por que foi escolhido ou deve ser desenvolvido este sistema

    Visto a ausência de alternativas que solucionem este problema em específico, o desenvolvimento de tal sistema mostra-se deveras útil para sanar esta lacuna da busca por objetos específicos na internet.

\begin{comment}
    Prof. Dr. Ausberto S. Castro Vera
    UENF - CCT - LCMAT - Curso de Ciência da Computação
    Campos, RJ, 2022
    Disciplina: Paradigma de Desenvolvimento Orientado a Objetos
    Aluno: João Vítor Fernandes Dias
\end{comment}